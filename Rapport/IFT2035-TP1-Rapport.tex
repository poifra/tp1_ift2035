%%%%%%%%%%%%%%%%%%%%%%%%%%%%%%%%%%%%%%%%%%%%%%%%%%%%%%%%%%%%%
%% Based on a TeXnicCenter-Template, which was             %%
%% created by Christoph Börensen                           %%
%% and slightly modified by Tino Weinkauf.                 %%
%%                                                         %%
%% Then, a third guy - me - put in some modifications.     %%
%%                                                         %%
%% IFT2035 - Rapport TP1                                   %%
%%%%%%%%%%%%%%%%%%%%%%%%%%%%%%%%%%%%%%%%%%%%%%%%%%%%%%%%%%%%%

\documentclass[letterpaper,12pt]{scrartcl}
% Optimised for letter. Add ",twosides" to use the two-sides layout.

% Margins
    \usepackage{vmargin}
    \setpapersize{USletter}
    \setmargins{2.0cm}%	 % Left edge
               {1.5cm}%  % Top edge
               {17.7cm}% % Text width
               {21.0cm}% % Text height
               {14pt}%	 % Header height
               {1cm}%    % Header distance
               {0pt}%	 % Footer height
               {2cm}%    % Footer distance
				
% Graphical bugfix (about footnotes)
    \usepackage[bottom]{footmisc}

% Fonts and locale
	\usepackage{t1enc}
	\usepackage[utf8]{inputenc}
	\usepackage{times}
	\usepackage[francais]{babel}
	\usepackage{SIunits}
	\usepackage{amsmath}

	\AtBeginDocument {%
	    \renewcommand\tablename{\textsc{Tableau}}
	}

% Graphics
	\usepackage[pdftex]{graphicx}
	\usepackage{color}
	\usepackage{eso-pic}
	\usepackage{everyshi}
	\renewcommand{\floatpagefraction}{0.7}

% Enable hyperlinks
	\usepackage[pdfborder=000,pdftex=true]{hyperref}
	
% Table layout
	\usepackage{booktabs}

% Caption
	\usepackage{ccaption}
	\captionnamefont{\bf\footnotesize\sffamily}
	\captiontitlefont{\footnotesize\sffamily}
	\setlength{\abovecaptionskip}{0mm}

% Header and footer settings
	\usepackage{scrpage2} 
	\renewcommand{\headfont}{\footnotesize\sffamily}
	\renewcommand{\pnumfont}{\footnotesize\sffamily}

% Pagestyles
	\defpagestyle{cb}{
		(\textwidth,0pt) % Sets the border line above the header
		{\pagemark\hfill\headmark\hfill} % Doublesided, left page
		{\hfill\headmark\hfill\pagemark} % Doublesided, right page
		{\hfill\headmark\hfill\pagemark} % Onesided
		(\textwidth,1pt)} % Sets the border line below the header
		{(\textwidth,1pt) % Sets the border line above the footer
		{{\it Rapport TP1 (IFT2035)}\hfill Sulliman Aïad et François Poitras} % Doublesided, left page
		{Sulliman Aïad et François Poitras\hfill{\it Rapport TP1 (IFT2035)}} % Doublesided, right page
		{Sulliman Aïad et François Poitras\hfill{\it Rapport TP1 (IFT2035)}} % One sided printing
		(\textwidth,0pt) % Sets the border line below the footer
	}

% Empty pages style
	\renewpagestyle{plain}
		{(\textwidth,0pt)
			{\hfill}{\hfill}{\hfill}
		(\textwidth,0pt)}
		{(\textwidth,0pt)
			{\hfill}{\hfill}{\hfill}
		(\textwidth,0pt)}

% Footnotes
	\renewcommand{\footnoterule}{\rule{5cm}{0.2mm} \vspace{0.3cm}}
	\deffootnote[1em]{1em}{1em}{\textsuperscript{\normalfont\thefootnotemark}}

\pagestyle{plain}

\begin{document}
	\begin{center}
		\vspace{2cm}

		{\Huge\bf\sf Rapport du Travail Pratique 1}

		\vspace{0.5cm}

		{\bf\sf (TP1)}

		\vspace{4cm}

		{\bf\sf Par}

		\vspace{0.5cm}{\large\bf\sf Sulliman Aïad et François Poitras}

		\vspace{2cm}

		{\bf\sf Rapport présenté à}

		\vspace{0.5cm}{\large\bf\sf M. Marc  Feeley}

		\vspace{2cm}

		{\bf\sf Dans le cadre du cours de}

		\vspace{0.5cm}{\large\bf\sf Concepts des langages de programmation (IFT2035)}

		\vspace{\fill}
		Remis le jeudi 22 octobre 2015

		\vspace{0.5cm}Université de Montréal
	\end{center}
	
	\newpage

	\pagestyle{cb}
	
	\tableofcontents

	\newpage
	
	\section{Fonctionnement du programme}
		 Une ou deux pages ici.
			
	%% END OF {Fonctionnement du programme} %%
	
    
	\section{Problèmes de programmation}
        Deux à quatre pages en tout.
        
		\subsection{Représentation des nombres et variables}
            
        
        \subsection{Analyse de l'entrée et calcul}
            
        
        \subsection{Gestion de la mémoire}
            
        
        \subsection{Algorithmes de calcul}
            
        
            \subsubsection{Addition}
                
            
            \subsubsection{Soustraction}
                
            
            \subsubsection{Multiplication}
                
            
        
        \subsection{Traitement des erreurs}
            
		
	%% END OF {Problèmes de programmation} %%
\end{document}